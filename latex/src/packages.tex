% Idioma: español
\usepackage[spanish,es-lcroman]{babel} % Características del idioma
\usepackage[utf8]{inputenc} % Acentos

%% Aspecto de página
%\usepackage[a4paper]{geometry} % Márgenes
%%\usepackage[left=4.72cm,right=4.72cm, top=4.4cm, bottom=4.21cm]{geometry} % Márgenes
%\usepackage{fancyhdr} % Headers
%\usepackage{parskip} % Espacio entre párrafos
%\usepackage{microtype} % Kerning y mejoras tipográficas
%\usepackage{pagecolor} % Color de fondo
%\usepackage{everypage} % \AddThispageHook {\tikz[remember picture,overlay] ... }
%
%% Columnas y tablas
%\usepackage{multicol}
%\usepackage{multirow}
%\usepackage{tabularx}
%\usepackage{tabu}
%\usepackage{diagbox} % Dividir una celda de un tabular
%\def\arraystretch{1}
%
%% Imágenes
\usepackage{graphicx}
%\usepackage{wrapfig}
\usepackage{caption}
\graphicspath{{img/}}
%
%% Símbolos
%\usepackage{mathtools} % \text, \overset, \underset
%\usepackage{amssymb} % \mathbb
%\usepackage{upgreek} % \uptheta
%\usepackage{amsthm} % QED
%\usepackage{marvosym} % \EUR
%
%% Formato
\usepackage{xcolor}
%\usepackage{soul} % \so, \st, \hl
%\usepackage{cancel} % \cancel, \bcancel, \xcancel, \cancelto
%\newcommand*\canceling[2][thin]{\tikz[baseline] \node [strike out,draw,anchor=text,inner sep=0pt,text=black,#1]{#2};}
%
%% Listas
%\usepackage{enumerate} % Numeración de listas
%\usepackage{outlines} % \0 \1 \2 \3 \4
%
%% Dibujos y gráficas
\usepackage{tikz}
%\usepackage{pgfplots}
\usetikzlibrary{shapes,arrows,arrows.meta,decorations.markings,shapes.misc,shapes.geometric,calc,pgfplots.groupplots}
%\pgfplotsset{compat=1.13}
%
%% Índices y títulos
%\addto\captionsspanish{\renewcommand{\contentsname}{Índice}}
%\usepackage{titlesec} % Formato de títulos
%%\usepackage{minitoc} % Índices de secciones
%\usepackage{hyperref} % Links
%\hypersetup{colorlinks}
%
%% Otros
%\usepackage[spanish,colorinlistoftodos,obeyDraft]{todonotes} % ToDo
%\usepackage{pdfpages} % Incluir PDFs
\usepackage{qrcode} % QR
%\usepackage{chemfig} % Química
%\usepackage{wasysym} % Notas musicales
%
%\usepackage[acronym]{glossaries} % Siglas y glosario
%\usepackage{acronym} % Acrónimos

% Código
\usepackage{listings}
\lstset{
	inputencoding=utf8,
	extendedchars=true,
	literate=%
	{á}{{\'a}}1
	{é}{{\'e}}1
	{í}{{\'i}}1
	{ó}{{\'o}}1
	{ú}{{\'u}}1
	{Á}{{\'A}}1
	{É}{{\'E}}1
	{Í}{{\'I}}1
	{Ó}{{\'O}}1
	{Ú}{{\'U}}1
	{ñ}{{\~n}}1
	{Ñ}{{\~N}}1
	{¿}{{>}}1
	{¡}{{<}}1
}
\lstset{ %
	backgroundcolor=\color{white},
	basicstyle=\footnotesize,
	breakatwhitespace=false,
	breaklines=true,
	extendedchars=true,              
	frame=single,          
	language=Python,                
	numbers=none,  
	rulecolor=\color{black},        
	showspaces=false,               
	showstringspaces=false,          
	showtabs=false,
	tabsize=4,
	morekeywords={},
	columns=flexible,    
	escapechar=¬,
}
%\lstset{
%language=[Latex]Tex,
%%
%basicstyle=\footnotesize\sffamily,
%keywordstyle=\footnotesize\sffamily,
%identifierstyle=\footnotesize\sffamily,
%commentstyle=\footnotesize\sffamily,
%stringstyle=\footnotesize\sffamily,
%escapechar=¬,
%%		
%numberstyle=\footnotesize\sffamily,%\ttfamily\tiny\color[gray]{0.3},
%numbers=left,
%stepnumber=2,
%numbersep=15pt,
%%
%columns=flexible,
%showstringspaces=false,
%tabsize=4,
%}

%% Apéndices
%\newcommand{\apendices}
%{\section*{Apéndices}%
%	\addcontentsline{toc}{section}{Apéndices}
%	\setcounter{section}{0}%
%	\setcounter{subsection}{0}%
%	\renewcommand\thesection{\Alph{section}}%
%}
%
%\newenvironment{changemargin}[2]{%
%	\begin{list}{}{%
%			\setlength{\topsep}{0pt}%
%			\setlength{\leftmargin}{#1}%
%			\setlength{\rightmargin}{#2}%
%			\setlength{\listparindent}{\parindent}%
%			\setlength{\itemindent}{\parindent}%
%			\setlength{\parsep}{\parskip}%
%		}%
%		\item[]
%}
%{\end{list}}